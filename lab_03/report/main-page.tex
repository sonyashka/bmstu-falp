\documentclass[14pt,a4paper]{scrreprt}

\include{preambule.inc}

\begin{document}

\include{00-title}

\thispagestyle{empty}

\chapter{Теоретические вопросы}

\section{Базис Lisp}

Базис языка --- минимальный набор конструкций языка и структур данных, позволяющих решить любую задачу.

К базису языка Lisp относятся:
\begin{itemize}
	\item атомы;
	\item бинарные узлы;
	\item встроенные функции --  atom, eq, car, cdr, cons;
	\item специальные функции и функционалы -- cond, quote, eval, lambda.
\end{itemize}

\section{Классификация функций}

Функции можно классифицировать с точки зрения организации.
\begin{enumerate}
	\item Чистые функции --- <<чистые математические>> функции (базис).
	\item Рекурсивные функции --- основной принцип организации повторных вычислений.
	\item Специальные функции или формы --- могут иметь переменное количество аргументов или вариативную их обработку.
	\item Псевдофункции --- создание каких-либо эффектов на экране.
	\item Функции с вариантами значений.
	\item Функционалы (функции высших порядков) --- в качестве аргументов используют функции или возвращают их в качестве результата.
	\item Базисные функции --- минимальный набор функций, позволяющих решить любую задачу.
\end{enumerate}

Также базисные и функции ядра можно классифицировать с точки зрения действий.
\begin{enumerate}
	\item Селекторы --- переходят по соответствующему указателю списковой ячейки.
	\item Конструкторы --- создают структуры данных.
	\item Предикаты --- позволяют классифицировать или сравнивать структуры.
\end{enumerate}

\section{Способы создания функций}

\begin{enumerate}
	\item С помощью lambda. После ключевого слова указывается лямбда-список и тело функции. 
	\begin{lstlisting}
	(lambda (x y) (+ x y))
	\end{lstlisting}
	Для применения используются лямбда-выражения.
	\begin{lstlisting}
	((lambda (x y) (+ x y)) 1 2)
	\end{lstlisting}
	\item С помощью defun. Используется для неоднократного применения функции (в том числе рекурсивного вызова).
	\begin{lstlisting}
	(defun sum (x y) (+ x y))
	(sum 1 2)
	\end{lstlisting}
\end{enumerate}

\section{Работа функций cond, if, and/or}

В языке Lisp для организации управления вычислительным процессом используются управляющие структуры. Внешне они схожи на вызов функции, отличие в использовании аргументов.\newpage

COND
\begin{lstlisting}
(cond (test1 value1)
	  (test2 value2)
	  ...
	  (testn valuen))
\end{lstlisting}

cond последовательно (слева направо) вычисляет значения testi (i=1..n), пока не получит значение, отличное от Nil. Как только оно встречено, результатом будет являтся valuei. Если все testi равны Nil, возвращает Nil.

IF
\begin{lstlisting}
(if test t_body f_body)
\end{lstlisting}

if вычисляет значение test; если оно отлично от Nil, выполняется t\_body, иначе -- f\_body.

AND/OR
\begin{lstlisting}
(and arg1 arg2 ... argn)
(or arg1 arg2 ... argn)
\end{lstlisting}

and последовательно вычисляет значения argi (i=1..n), пока не встретится Nil. Если такое значение встретилось, то функция возвращает Nil, иначе -- последнее вычисленное значение.

or последовательно вычисляет значение argi (i=1..n), пока не встретится значение, отличное от Nil. Если такое встретилос, то функция возвращает последнее вычисленное значение, иначе -- Nil.

\end{document}