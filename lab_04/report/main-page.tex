\documentclass[14pt,a4paper]{scrreprt}

\include{preambule.inc}

\begin{document}

\include{00-title}

\thispagestyle{empty}

\chapter{Теоретические вопросы}

\section{Синтаксическая форма и хранение программы в памяти}

В Lisp программа синтаксически представлена в форме S-выражений. Особенностью является единая форма фиксации (отсутствие разделения на программу и данные). И то, и другое представляется списочной структурой, имеющей одинаковую форму. Благодаря такому подходу возможно изменение кода программы при обработке данных.

Так как программа имеет вид S-выражения, в памяти она представлена либо как атом (5 указателей, которыми представляется атом в памяти), либо как списковая ячейка (2 указателя, бинарный узел).

\section{Трактовка элементов списка}

При обработке списков первый элемент воспринимается интерпретатором как название функции, все остальные -- ее аргументы. Количество элементов, не считая первого -- названия функции, должно совпадать с количеством входных аргументов указанной функции.

В случае если перед скобкой стоит блокировка (` или '), вычисления не производятся, и результатом является все, что стоит после блокировки.

\begin{lstlisting}
	(defun mult (a b c) (* a b c))
	(mult 1 2 3) ; => 6; mult - name of func; 1, 2, 3 - arguments
	(mult 0 1 2 3) ; => error - invalid number of arguments
	'(eval func 3) ; => (eval func 3)
\end{lstlisting}
\newpage

\section{Порядок реализации программы}

\begin{enumerate}
	\item Ожидает ввода S-выражения.
	\item Передает введенное S-выражение функции eval.
	\item Выводит полученный результат.
\end{enumerate}

\begin{figure}[H]
	\begin{center}
		\includegraphics[scale=0.5]{assets/scheme.png}
	\end{center}
	\caption{Диаграмма работы функции eval}
\end{figure}


\section{Способы определения функции}

\begin{enumerate}
	\item С помощью lambda. После ключевого слова указывается лямбда-список и тело функции. 
	\begin{lstlisting}
	(lambda (x y) (+ x y))
	\end{lstlisting}
	Для применения используются лямбда-выражения.
	\begin{lstlisting}
	((lambda (x y) (+ x y)) 1 2)
	\end{lstlisting}
	\item С помощью defun. Используется для неоднократного применения функции (в том числе рекурсивного вызова).
	\begin{lstlisting}
	(defun sum (x y) (+ x y))
	(sum 1 2)
	\end{lstlisting}
\end{enumerate}

\end{document}