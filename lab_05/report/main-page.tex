\documentclass[14pt,a4paper]{scrreprt}

\include{preambule.inc}

\begin{document}

\include{00-title}

\thispagestyle{empty}

\chapter{Теоретические вопросы}

\section{Структуроразрушающие и не разрушающие структуру списка функции}

Функции работающие со списками делятся на две группы.
\begin{enumerate}
	\item Не разрушающие структуру. Данный тип создает копии всех аргументов, а именно их списковых ячеек (не car-указателей), и расставляет cdr-указатели на новые. Копия последнего аргумента не создается для оптимизации работы по времени. Можно отнести append, reverse, conc.
	\item Разрушающие структуру. Не создает копий, а переставляет значения cdr-указателей исходных списковых ячеек. Названия данных функций начинаются с n: nconc, nreverse.
\end{enumerate}

\section{Отличие в работе cons, list, append, nconc и в их результате}

CONS
\begin{itemize}
	\item входными параметрами являются 2 S-выражения;
	\item создает списковую ячейку, расставляя car- и cdr-указатели на соответствующие аргументы;
	\item результатом является точечная пара.
\end{itemize}

LIST
\begin{itemize}
	\item неограниченное число входных параметров -- S-выражений;
	\item создает список, где количество списковых ячеек равно количеству аргументов, и расставляет car-указатели на аргументы;
	\item результатом является список.
\end{itemize}

APPEND
\begin{itemize}
	\item неограниченное число входных параметров -- списков;
	\item создает копии всех аргументов кроме последнего (только списковых ячеек, расставляя соответствующие car-указатели) и расставляет cdr-указатели на головы списков;
	\item результатом является список из копий аргументов кроме последнего. При этом изменение последнего повлечет за собой изменение исходных данных.
\end{itemize}

NCONC
\begin{itemize}
	\item неограниченное число входных параметров -- списков;
	\item переставляет последние cdr-указатели аргументов на голову следующего списка-аргумента;
	\item результатом является список, состоящий из исходных списковых ячеек.
\end{itemize}

\end{document}