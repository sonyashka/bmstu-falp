\documentclass[14pt,a4paper]{scrreprt}

\include{preambule.inc}

\begin{document}

\include{00-title}

\thispagestyle{empty}

\chapter{Часть 1}

\section{Задание}

Запустить среду Visual Prolog5.2. Настроить утилиту TestGoal (способ настройки см. в
дополнительных материалах к лаб. раб.). Запустить тестовую программу, проанализировать
реакцию системы и множество ответов. Разработать свою программу - «Телефонный
справочник». Протестировать работу программы.

\section{Ответы на вопросы}

\begin{itemize}
	\item Что собой представляет программа на Prolog?\newlineПрограмма на Prolog представляет собой набор фактов и правил, обеспечивающих получение заключений на
	основе этих утверждений. Программа содержит базу знаний и вопрос. База знаний содержит истинностные знания, используя которые программа выдает ответ на запрос.
	\item Какова ее структура?\newlineПрограмма состоит из следующих разделов (наличие всех является необязательным):
	\begin{itemize}
		\item директивы компилятора -- зарезервированные символьные константы;
		\item CONSTANTS -- раздел описания констант;
		\item DOMAINS -- раздел описания доменов;
		\item DATABASE -- раздел описания предикатов внутренней базы данных;
		\item PREDICATES -- раздел описания предикатов;
		\item CLAUSES -- раздел описания предложений базы знаний;
		\item GOAL -- раздел описания внутренней цели (задания вопроса);
	\end{itemize}
	\item Как программа реализуется? Как формируются результаты работы?\newlineЦелью программы является ответ на заданный вопрос, который дается в логической форме -- <<Да>> или <<Нет>>. При этом система пытается подобрать такие комбинации из базы знаний, чтобы ответить <<Да>>, подходящих комбинаций может быть несколько. Система использует механизм унификации -- операция, которая позволяет формализовать процесс логического вывода. С ее помощью происходит:
	\begin{itemize}
		\item двунаправленная передача параметров процедурам;
		\item неразрушающее присваивание;
		\item проверка условий.
	\end{itemize}
	В ходе работы выполняется большое число унификаций. Попытка сопоставления двух термов может закончится успехом или неудачей; в случае последней происходит возвращение к предыдущему шагу.
\end{itemize}

\chapter{Часть 2}

\section{Задание}

Составить программу – базу знаний, с помощью которой можно определить, например,
множество студентов, обучающихся в одном ВУЗе и их телефоны. Студент может
одновременно обучаться в нескольких ВУЗах. Привести примеры возможных вариантов
вопросов и варианты ответов (не менее 3-х). Описать порядок формирования вариантов
ответа. Исходную базу знаний сформировать с помощью только фактов.

\section{Ответы на вопросы}

\begin{itemize}
	\item Назначение и использование переменных.\newlineПеременные предназначены для обозначения неизвестного объекта в предметной области (т.е. служат для повышения уровня абстракции), им не задаются значения в коде программы, система самостоятельно подбирает такие, чтобы на поставленный вопрос суметь ответить <<Да>>. Различают два вида переменных:
	\begin{itemize}
		\item именованные (обозначаются комбинацией символов, начиная с большой латинской буквы или символа <<\_>>) -- являются уникальными в рамках предложения, необходимы для передачи данных во времени и в пространстве;
		\item анонимные (обозначаются символом <<\_>>) -- являются уникальными в рамках всей программы.
	\end{itemize} 
	\item Порядок формирования результата работы программы (ответ дан в предыдущей части).
	\item Что собой представляет программа на Prolog, какова ее
	структура, как она реализуется? (ответ дан в предыдущей части)
\end{itemize}

\end{document}