\documentclass[14pt,a4paper]{scrreprt}

\include{preambule.inc}

\begin{document}

\include{00-title}

\thispagestyle{empty}

\chapter{Теоретические вопросы}

\section{Базис Lisp}

Базис языка --- минимальный набор конструкций языка и структур данных, позволяющих решить любую задачу.

К базису языка Lisp относятся:
\begin{itemize}
	\item атомы;
	\item бинарные узлы;
	\item встроенные функции --  atom, eq, car, cdr, cons;
	\item специальные функции и функционалы -- cond, quote, eval, lambda.
\end{itemize}

\section{Классификация функций}

Функции можно классифицировать с точки зрения организации.
\begin{enumerate}
	\item Чистые функции --- <<чистые математические>> функции (базис).
	\item Рекурсивные функции --- основной принцип организации повторных вычислений.
	\item Специальные функции или формы --- могут иметь переменное количество аргументов или вариативную их обработку.
	\item Псевдофункции --- создание каких-либо эффектов на экране.
	\item Функции с вариантами значений.
	\item Функционалы (функции высших порядков) --- в качестве аргументов используют функции или возвращают их в качестве результата.
	\item Базисные функции --- минимальный набор функций, позволяющих решить любую задачу.
\end{enumerate}

Также базисные и функции ядра можно классифицировать с точки зрения действий.
\begin{enumerate}
	\item Селекторы --- переходят по соответствующему указателю списковой ячейки.
	\item Конструкторы --- создают структуры данных.
	\item Предикаты --- позволяют классифицировать или сравнивать структуры.
\end{enumerate}

\section{Способы создания функций}

\begin{enumerate}
	\item С помощью lambda. После ключевого слова указывается лямбда-список и тело функции. 
	\begin{lstlisting}
	(lambda (x y) (+ x y))
	\end{lstlisting}
	Для применения используются лямбда-выражения.
	\begin{lstlisting}
	((lambda (x y) (+ x y)) 1 2)
	\end{lstlisting}
	\item С помощью defun. Используется для неоднократного применения функции (в том числе рекурсивного вызова).
	\begin{lstlisting}
	(defun sum (x y) (+ x y))
	(sum 1 2)
	\end{lstlisting}
\end{enumerate}

\section{Функции Car и Cdr}

Функции Car и Cdr являются базовыми функциями работы со списками.
В качестве аргументов им передается точечная пара или список.

Car возвращает голову (первый элемент) списка, Cdr -- хвост (все элементы, кроме первого).

Если входной аргумент является пустым списком, обе функции возвращают Nil. Если в списке 1 элемент, Car вернет голову, а Cdr -- Nil.

\section{Назначение и отличие в работе Cons и List}

Cons --- функция от двух аргументов. Создает списковую ячейку и расставляет 2 указателя -- на голову и на хвост -- на входные аргументы.

List --- функция от произвольного числа аргументов, при этом все они вычисляются. Строит новый список, первым элементом которого является значение первого аргумента, хвостом -- значение второго аргумента.

\begin{lstlisting}
(cons '(A) '(B)) ;; ((A) B)
(list '(A) '(B)) ;; ((A) (B))
\end{lstlisting}

\end{document}