\documentclass[14pt,a4paper]{scrreprt}

\include{preambule.inc}

\begin{document}

\include{00-title}

\thispagestyle{empty}

\chapter{Практическое задание}

\section{Задание}

Создать базу знаний <<Предки>>, позволяющую наиболее эффективным способом (за меньшее количество шагов, что обеспечивается меньшим количеством предложений в БЗ -- правил), и используя разные варианты (примеры) одного вопроса, определить (указать, какой вопрос для какого варианта):
\begin{enumerate}
	\item по имени субъекта определить всех его бабушек (предки 2-го колена);
	\item по имени субъекта определить всех его дедушек (предки 2-го колена);
	\item по имени субъекта определить всех его бабушек и дедушек (предки 2-го колена);
	\item по имени субъекта определить его бабушку по материнской линии (предки 2-го колена);
	\item по имени субъекта определить его бабушку и дедушку по материнской линии (предки 2-го колена).
\end{enumerate}
Минимизировать количество правил и количество вариантов вопросов. Использовать конъюнктивные правила и простой вопрос.

Для одного из вариантов вопроса и конкретной БЗ составить таблицу, отражающую конкретный порядок работы системы, с объяснениями:
\begin{itemize}
	\item очередная проблема на каждом шаге и метод ее решения;
	\item каково новое текущее состояние резольвенты, как получено;
	\item какие дальнейшие действий? (Запускается ли алгоритм унификации? Каких термов? Почему этих?);
	\item вывод по результатам очередного шага и дальнейшие действия.
\end{itemize}
Т.к. резольвента хранится в виде стека, то состояние резольвенты требуется отображать в столбик: вершина -- сверху! Новый шаг надо начинать с нового состояния резольвенты!

\section{Код программы}

\begin{lstlisting}[language=Prolog]
domains

	name, gender = symbol.
	human = human(name, gender).

predicates

	isParent(human, human).
	isGrandparent(human, gender, human).

clauses

	isGrandparent(human(GrandName, GrandGender), LineGender, 
		human(Name, Gender)) :- 
			isParent(human(GrandName, GrandGender), human(TmpName, 
			LineGender)), 
			isParent(human(TmpName, LineGender), human(Name, Gender)).
	
	isParent(human("Irina", woman), human("Olga", woman)).
	isParent(human("Igor", man), human("Olga", woman)).
	isParent(human("Victor", man), human("Irina", woman)).
	isParent(human("Svetlana", woman), human("Irina", woman)).
	isParent(human("Fedor", man), human("Igor", man)).
	isParent(human("Uliya", woman), human("Igor", man)).

goal
	%isGrandparent(human(Name, woman), _, human("Olga", woman)).
	%isGrandparent(human(Name, man), _, human("Olga", woman)).
	%isGrandparent(Human, _, human("Olga", woman)).
	%isGrandparent(human(Name, woman), woman, human("Olga", woman)).
	isGrandparent(Human, woman, human("Olga", woman)).
\end{lstlisting}

\newpage
\section{Таблицы выполнения программы}

Запрос для задания 2:

isGrandparent(human(Name, man), \_, human(``Olga'', woman)).


\begin{table}[H]
	\begin{tabular}{|p{0.8cm\small}|p{4.7cm\small}|p{5.7cm\small}|p{4cm\small}|}	
		\hline
		№ шага & Состояния резольвенты и вывод: дальнейшие действия (почему?) & Для каких термов запускается алгоритм унификации: T1=T2 и каков результат (подстановка) & Дальнейшие действия: прямой ход или откат (почему и к чему приводит?)\\
		\hline
		1 & Резольвента:\begin{itemize}
			\item isParent(human(
		TmpName, \_), human(``Olga'', woman)) \item isParent(human(
		Name, man), human(TmpName, \_))\end{itemize} & isGrandparent(human(Name, man), \_, human(``Olga'', woman)) = isGrandparent(human(
		GrandName, GrandGender), LineGender, human(Name, Gender)). Унификация успешна\linebreak \textbf{Подстановка:} \{GrandName=Name, GrandGender=man, Name=``Olga'', Gender=woman\} & Прямой ход, приводит к попытке унификации isParent(human(
		TmpName, \_), human(``Olga'', woman)) и isGrandparent(human(
		GrandName, GrandGender), LineGender, human(Name, Gender))\\
		\hline 
		2 & Резольвента:\begin{itemize} \item isParent(human(
			TmpName, \_), human(``Olga'', woman)) \item isParent(human(
			Name, man), human(TmpName, \_))\end{itemize} & isParent(human(TmpName, \_), human(``Olga'', woman)) = isGrandparent(human(
		GrandName, GrandGender), LineGender, human(Name, Gender)). Унификация неуспешна & Прямой ход, переход к следующему предложению\\
		\hline
		3 & Резольвента:\begin{itemize} \item isParent(human(
			Name, man), human(``Irina'', \_))\end{itemize} & isParent(human(TmpName, \_), human(``Olga'', woman)) = isParent(human(``Irina'', woman), human(``Olga'', woman)). Унификация успешна\linebreak
		\textbf{Подстановка:} \{GrandName=Name, GrandGender=man, Name=``Olga'', Gender=woman, TempName=``Irina''\} & Прямой ход, попытка унификации isParent(human(
		Name, man), human(``Irina'', \_))\\
		\hline
	\end{tabular}
\end{table}

\begin{table}[H]
	\begin{tabular}{|p{0.8cm\small}|p{4.7cm\small}|p{5.7cm\small}|p{4cm\small}|}	
		\hline
		№ шага & Состояния резольвенты и вывод: дальнейшие действия (почему?) & Для каких термов запускается алгоритм унификации: T1=T2 и каков результат (подстановка) & Дальнейшие действия: прямой ход или откат (почему и к чему приводит?)\\
		\hline
		4 & Резольвента:\begin{itemize} \item isParent(human(
			Name, man), human(``Irina'', \_))\end{itemize} & isParent(human(
		Name, man), human(``Irina'', \_)) = isGrandparent(human(
		GrandName, GrandGender), LineGender, human(Name, Gender)). Унификация неуспешна & Прямой ход, переход к следующему предложению\\
		\hline
		5-6 & ... & Унификация неуспешна & ...\\
		\hline
		7 & Резольвента: пуста\linebreak \textbf{Вывод:} Name=``Victor'' & isParent(human(
		Name, man), human(``Irina'', \_)) = isParent(human(``Victor'', man), human(``Irina'', woman)). Унификация успешна\linebreak \textbf{Подстановка:} \{GrandName=``Victor'', GrandGender=man, Name=``Olga'', Gender=woman, TempName=``Irina''\} & Откат, переход к следующему предложению\\
		\hline
		8-9 & Резольвента: \begin{itemize} \item isParent(human(
			Name, man), human(``Irina'', \_))\end{itemize} & Унификация неуспешна & Прямой ход, переход к следующему предложению\\
		\hline
		10 & Резольвента:\begin{itemize} \item isParent(human(
			TmpName, \_), human(``Olga'', woman)) \item isParent(human(
			Name, man), human(TmpName, \_))\end{itemize} & isParent(human(
		Name, man), human(``Irina'', \_)) = isParent(human(``Uliya'', woman), human(``Igor'', man)). Унификация неуспешна & Откат, достижение конца БЗ, переход к следющему предложению относительно 3\\
		\hline
		11 & Резольвента:\begin{itemize} \item isParent(human(
			Name, man), human(``Igor'', \_))\end{itemize} & isParent(human(TmpName, \_), human(``Olga'', woman)) = isParent(human(``Igor'', man), human(``Olga'', woman)). Унификация успешна\linebreak
		\textbf{Подстановка:} \{GrandName=Name, GrandGender=man, Name=``Olga'', Gender=woman, TempName=``Igor''\} & Прямой ход, попытка унификации isParent(human(
		Name, man), human(``Igor'', \_))\\
		\hline
	\end{tabular}
\end{table}

\begin{table}[H]
	\begin{tabular}{|p{0.8cm\small}|p{4.7cm\small}|p{5.7cm\small}|p{4cm\small}|}	
		\hline
		№ шага & Состояния резольвенты и вывод: дальнейшие действия (почему?) & Для каких термов запускается алгоритм унификации: T1=T2 и каков результат (подстановка) & Дальнейшие действия: прямой ход или откат (почему и к чему приводит?)\\
		\hline
		12 & Резольвента:\begin{itemize} \item isParent(human(
			Name, man), human(``Igor'', \_))\end{itemize} & isParent(human(
		Name, man), human(``Igor'', \_)) = isGrandparent(human(
		GrandName, GrandGender), LineGender, human(Name, Gender)). Унификация неуспешна & Прямой ход, переход к следующему предложению\\
		\hline
		13-17 & ... & Унификация неуспешна & ... \\
		\hline
		18 & Резольвента: пуста\linebreak \textbf{Вывод:} Name=``Fedor'' & isParent(human(
		Name, man), human(``Igor'', \_)) = isParent(human(``Fedor'', man), human(``Igor'', man)). Унификация успешна\linebreak \textbf{Подстановка:} \{GrandName=``Fedor'', GrandGender=man, Name=``Olga'', Gender=woman, TempName=``Igor''\} & Откат, переход к следующему предложению\\
		\hline
		19 & Резольвента:\begin{itemize} \item isParent(human(
			Name, man), human(``Igor'', \_))\end{itemize} & isParent(human(
		Name, man), human(``Igor'', \_)) = isParent(human(``Uliya'', woman), human(``Igor'', man)). Унификация неуспешна & Откат, достижение конца БЗ, переход к следующему предложению относительно 11\\
		\hline
		20 & Резольвента:\begin{itemize} \item isParent(human(
			TmpName, \_), human(``Olga'', woman)) \item isParent(human(
			Name, man), human(TmpName, \_))\end{itemize} & isParent(human(TmpName, \_), human(``Olga'', woman)) = isParent(human(``Victor'', man), human(``Irina'', woman)). Унификация неуспешна & Прямой ход, переход к следующему предложению\\
		\hline
		21-23 & ... & Унификация неуспешна & Откат, достижение конца БЗ, переход к следующему предложению относительно 1\\
		\hline
		24-29 & Резольвента: пуста & isGrandparent(human(Name, man), \_, human(``Olga'', woman)) = isParent(...). Унификация неуспешна & Откат, достижение конца БЗ, завершение работы\\
		\hline
	\end{tabular}
\end{table}
		
\end{document}