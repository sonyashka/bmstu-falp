\documentclass[14pt,a4paper]{scrreprt}

\include{preambule.inc}

\begin{document}

\include{00-title}

\thispagestyle{empty}

\chapter{Теоретические вопросы}

\section{Элементы языка: определение, синтаксис, представление в памяти}

Элементы языка --- атомы и точечные пары (структуры, которые строятся с помощью унифицированных структур - блоков памяти - бинарных узлов). Атомы бывают:
\begin{itemize}
	\item \textbf{символы} (идентификаторы) --- синтаксически представляют собой набор литер (последовательность букв и цифр, начинающаяся с буквы; могут быть связанные и несвязанные);
	\item \textbf{специальные символы} --- используются для обозначения <<логических>> констант (\texttt{T}, \texttt{Nil});
	\item \textbf{самоопределимые атомы} --- числа, строки - последовательность символов в кавычках (\texttt{"abc"}).
\end{itemize}

\section{Синтаксис элементов языка}

\texttt{Точечная пара ::= (<атом> . <атом>) | (<точечная пара> . <атом>) | (<атом> . <точечная пара>) | (<точечная пара> . <точечная пара>)}

\texttt{Список ::= <пустой список> | <непустой список>}, где 

\texttt{<пустой список> ::= () | Nil},

\texttt{<непустой список> ::= (<S-выражение>. <список>)},

Список --- частный случай S-выражения.

Синтаксически любая структура (точечная пара или список) заключается в круглые скобки:
\texttt{(A . B)} --- точечная пара.
\texttt{(A)} --- список из одного элемента.
Непустой список --- \texttt{(A . (B . (C . (D . Nil))))} или \texttt{(A B C D)}
Пустой список --- \texttt{Nil} или \texttt{()}.

Элементы списка могут быть списками, например --- \texttt((A (B C) (D (E)))). Таким образом, синтаксически наличие скобок является признаком структуры --- списка или точечной пары.

Любая непустая структура Lisp в памяти представляется  списковой ячейкой, хранящий два указателя: на голову (первый элемент) и хвост (все остальное).
 

\section{Особенности языка Lisp. Структура программы. Символ апостроф}

Можно выделить следующие особенности языка Lisp.
\begin{enumerate}
	\item Использование символьной обработки.
	\item Программа может быть представлена в виде данных, что позволяет ей самой изменять себя.
	\item Отсутствие типизации (типов данных).
	\item Блочное выделение памяти, этим занимается сам Lisp.
	\item Программы представляются в виде списков.
\end{enumerate}

Программа на языке Lisp записывается в виде последовательности атомов и списков, т.е. программа и данные имеют одинаковый синтаксис. Другими словами программа представляет собой последовательность форм, а форма или вычислимое выражение -- это атом или список, который можно вычислить и получить значение. 
Как спецальная функция \texttt{quote}. Данная функция блокирует вычисления своего единственного аргумента, то есть он воспринимается как константа. При выполнении функции аргумент обрабатывается по общей схеме.

\section{Базис языка Lisp. Ядро языка}


\begin{itemize}
	\item car -- возвращает первый элемент списка, являющегося значением ее единственного аргумента;
	\item cdr -- возвращает хвост списка, являющегося значением его ее единственного аргумента;
	\item cons --  строит новый список, первым элементом которого является первый аргумент, хвостом -- второй;
	\item atom -- возвращает значение T, если аргумент является атомом, иначе возвращает Nil;
	\item eq -- проверяет совпадение двух своих аргументов-атомов, возвращает значение T если значение одного из его аргументов -- атом и одновременно значения аргументов эквивалентны;
	\item quote -- в качестве значения выдает аргумент, не вычисляя его значения;
	\item eval -- выполняет двойное вычисление своего аргумента, чаще всего применяется для снятия блокировки функцией quote;
	\item cond -- условное выражение, записывается как (cond ($p_1 e_1)..(p_n e_n$)), где $(p_i e_i)$ -- ветви условного выражения, выражения-формы $p_i$ -- условия ветвей.
\end{itemize}



\begin{itemize}
	\item caar, cadr, cdar, cddr, caaar, ..., cdddar, cddddr -- функции от одного аргумента, являющиеся суперпозицией функций car и cdr;
	\item list -- составляет список из значений своих аргументов (в отличие от cons является симметричной относительно своих аргументов);
	\item +, -, *, / -- простые арифметические функции;
	\item =, /=, <, >, <=, >= -- арифметические предикаты;
	\item evenp -- проверка числа на четность;
	\item null -- проверка атома на равенство Nil;
	\item listp -- проверка на список;
	\item numberp и symbolp -- проверка на числовой или символьный атом соответственно;
	\item not, and, or -- логические отрицание, конъюнкция, дизъюнкция.
\end{itemize}

\end{document}