\documentclass[14pt,a4paper]{scrreprt}

\include{preambule.inc}

\begin{document}

\include{00-title}

\thispagestyle{empty}

\chapter{Практическое задание}

\section{Задание}

Создать базу знаний <<Собственники>>, дополнив (и минимально изменив) базу знаний, хранящую:
\begin{itemize}
	\item <<Телефонный справочник>>: Фамилия, №тел, Адрес -- структура (Город, Улица, №дома, №кв.);
	\item <<Автомобили>>: Фамилия\_владельца, Марка, Цвет, Стоимость, и др.;
	\item <<Вкладчики банков>>: Фамилия, Банк, счет, сумма, др.;
\end{itemize}
знаниями о дополнительной собственности владельца. Преобразовать знания об автомобиле к форме знаний о собственности. Виды собственности (кроме автомобиля):
\begin{itemize}
	\item Строение, стоимость и другие его характеристики;
	\item Участок, стоимость и другие его характеристики;
	\item Водный транспорт, стоимость и другие его характеристики.
\end{itemize}
Описать и использовать вариантный домен: Собственность. Владелец может иметь, но только один объект каждого вида собственности (касается и автомобиля), или не иметь некоторых видов собственности.

Используя конъюнктивное правило и разные формы задания одного вопроса (пояснить для какого № задания -- какой вопрос), обеспечить возможность поиска:
\begin{enumerate}
	\item Названий всех объектов собственности заданного субъекта;
	\item Названий и стоимости всех объектов собственности заданного субъекта;
	\item * Разработать правило, позволяющее найти суммарную стоимость всех объектов собственности заданного субъекта.
\end{enumerate}
Для 2-го пункта и одной фамилии составить таблицу, отражающую конкретный порядок работы системы, с объяснением порядка работы и особенностей использования доменов (указать конкретные T1 и T2 и полную подстановку на каждом шаге).

\section{Код программы}

\begin{lstlisting}[language=Prolog]
domains    

	surname, phone, city, street = symbol.    
	homeNumber, appartmentNumber = unsigned.
	address = address(city, street, homeNumber, appartmentNumber). 
	
	model, color = symbol.   
	price = unsigned.      
	
	bank, account = symbol.   
	sum = unsigned.    
	
	type = symbol.   
	size = unsigned.      

	ownership = building(address, price); 
	area(size, price); 
	waterTrasnport(type, color, price); 
	car(model, color, price).  

predicates    

	hasPhone(surname, phone, address).   
	hasDeposit(surname, bank, account, sum).      
	
	own(surname, ownership).   
	ownObject(surname, symbol, price).
	ownObjectPass(surname, symbol, price).  
	objectsPrice(surname, price).

clauses    
	hasPhone("Balashov", "+79741632985", address("Moscow", "Baumanskaya", 
		15, 21)).   
	hasPhone("Serov", "+79146941728", address("Lipetsk", "Gagarina", 192, 
		13)).   
	hasPhone("Paraskun", "+79172641928", address("Moscow", "Izmaylovskaya",
		73, 2)).    
	hasDeposit("Balashov", "Home-credit", "5148465849516259", 24318947).   
	hasDeposit("Balashov", "VTB", "5670148746192648", 478976).   
	hasDeposit("Paraskun", "Sberbank", "7193019871942510", 100000).
	
	own("Balashov", area(20, 139200)).   
	own("Balashov", car("BMW-Y15", "Red", 2345700)).   
	own("Paraskun", building(address("Moscow", "Lubyanka", 13, 182),
		1410000)).   
	own("Paraskun", waterTrasnport("Bike", "White", 80000)).
	
	ownObject(Surname, building, Price) :- own(Surname, building(_, Price)).
	ownObject(Surname, area, Price) :- own(Surname, area(_, Price)).
	ownObject(Surname, waterTrasnport, Price) :- own(Surname, 
		waterTrasnport(_, _, Price)).
	ownObject(Surname, car, Price) :- own(Surname, car(_, _, Price)).
	
	ownObjectPass(Surname, building, Price) :- own(Surname, building(_, 
		Price)), !.
	ownObjectPass(Surname, area, Price) :- own(Surname, area(_, Price)), !.
	ownObjectPass(Surname, waterTrasnport, Price) :- own(Surname, 
		waterTrasnport(_, _, Price)), !.
	ownObjectPass(Surname, car, Price) :- own(Surname, car(_, _, Price)), !.
	ownObjectPass(_, _, 0).
	
	objectsPrice(Surname, Price) :- 
		ownObjectPass(Surname, building, BPrice),
		ownObjectPass(Surname, area, APrice),
		ownObjectPass(Surname, waterTransport, WtPrice),
		ownObjectPass(Surname, car, CPrice),
		Price = BPrice + APrice + WtPrice + CPrice.

goal
	%ownObject("Balashov", Objects, _).
	%ownObject("Paraskun", Objects, Price).
	objectsPrice("Paraskun", Price).
\end{lstlisting}

\section{Таблицы выполнения программы}

Запрос для заданий 2:

ownObject(``Paraskun'', Objects, Price).

\begin{table}[H]
	\begin{tabular}{|p{1.2cm\small}|p{9cm\small}|p{5cm\small}|}	
		\hline
		№ шага & Сравниваемые термы; результат; подстановка, если есть & Дальнейшие действия: прямой ход или откат (к чему приводит?)\\
		\hline
		1 & Сравнение:\linebreak ownObject(``Paraskun'', Objects, Price) = hasPhone(``Balashov'', ``+79741632985'', address(``Moscow'', ``Baumanskaya'', 15, 21)). Унификация неуспешна (несовпадение функторов) & Прямой ход, переход к следующему предложению\\
		\hline
		2-10 & ... & ...\\
		\hline
		11 & Сравнение:\linebreak ownObject(``Paraskun'', Objects, Price) = ownObject(Surname, building, Price). Унификация успешна.\linebreak \textbf{Подстановка:} \{Surname=``Paraskun'', Objects=building, Price=Price\} & Новое состояние резольвенты: own(``Paraskun'', building(\_, Price))\\
		\hline
		12 & Сравнение:\linebreak own(``Paraskun'', building(\_, Price)) = hasPhone(``Balashov'', ``+79741632985'', address(``Moscow'', ``Baumanskaya'', 15, 21)). Унификация неуспешна (несовпадение функторов) & Прямой ход, переход к следующему предложению\\
		\hline
		13-19 & Унификация неуспешна & Прямой ход, переход к следующему предложению\\
		\hline
		20 & Сравнение:\linebreak own(``Paraskun'', building(\_, Price)) = own(``Paraskun'', building(address(``Moscow'', ``Lubyanka'', 13, 182), 1410000)). Унификация успешна.\linebreak \textbf{Подстановка:} \{Surname=``Paraskun'', Objects=building, Price=1410000\} & Новое состояние резольвенты: пуста\linebreak \textbf{Вывод:} Objects=building, Price=1410000\linebreak Откат, следующее предложение, \textbf{новая подстановка:} \{Surname=``Paraskun'', Objects=building, Price=Price\}\\
		\hline
		21 & Сравнение:\linebreak own(``Paraskun'', building(\_, Price)) = own(``Paraskun'', waterTrasnport(``Bike'', ``White'', 80000)). Унификация неуспешна (несовпадение термов) & Прямой ход, переход к следующему предложению\\
		\hline
		22-31 & ... (несовпадение функторов) & Откат, достижение конца БЗ, переход к следующему предложению относительно 11\\
		\hline
		32 & Сравнение:\linebreak ownObject(``Paraskun'', Objects, Price) = ownObject(Surname, area, Price). Унификация успешна.\linebreak \textbf{Подстановка:} \{Surname=``Paraskun'', Objects=area, Price=Price\} & Новое состояние резольвенты: own(``Paraskun'', area(\_, Price))\\
		\hline
		33-53 & Унификация неуспешна & Откат, достижение конца БЗ, переход к следующему предложению относительно 32\\
		\hline
	\end{tabular}
\end{table}

\begin{table}[H]
	\begin{tabular}{|p{1.2cm\small}|p{9cm\small}|p{5cm\small}|}	
		\hline
		№ шага & Сравниваемые термы; результат; подстановка, если есть & Дальнейшие действия: прямой ход или откат (к чему приводит?)\\
		\hline
		54 & Сравнение:\linebreak ownObject(``Paraskun'', Objects, Price) = ownObject(Surname, waterTransport, Price). Унификация успешна.\linebreak \textbf{Подстановка:} \{Surname=``Paraskun'', Objects=waterTransport, Price=Price\} & Новое состояние резольвенты: own(``Paraskun'', waterTransport(\_, \_, Price))\\
		\hline
		55-63 & Унификация неуспешна & Прямой ход, переход к следующему предложению\\
		\hline
		64 & Сравнение:\linebreak ownObject(``Paraskun'', Objects, Price) = own(``Paraskun'', waterTrasnport(``Bike'', ``White'', 80000)). Унификация успешна.\linebreak \textbf{Подстановка:} \{Surname=``Paraskun'', Objects=waterTransport, Price=80000\} & Новое состояние резольвенты: пуста\linebreak \textbf{Вывод:} Objects=waterTransport, Price=80000\linebreak Откат, переход к следующему предложению, \textbf{новая подстановка:} \{Surname=``Paraskun'', Objects=waterTransport, Price=Price\}\\
		\hline
		65-74 & Унификация неуспешна & Откат, достижение конца БЗ, переход к следующему предложению относительно 54\\
		\hline
		75 & Сравнение:\linebreak ownObject(``Paraskun'', Objects, Price) = ownObject(Surname, car, Price). Унификация успешна.\linebreak \textbf{Подстановка:} \{Surname=``Paraskun'', Objects=car, Price=Price\} & Новое состояние резольвенты: own(``Paraskun'', car(\_, \_, Price))\\
		\hline
		76-95 & Унификация неуспешна & Откат, достижение конца БЗ, переход к следующему предложению относительно 75\\
		\hline
		96 & Сравнение:\linebreak ownObject(``Paraskun'', Objects, Price) = ownObjectPass(Surname, building, Price).Унификация неуспешн (несовпадение функторов) & Прямой ход, переход к следующему предложению\\
		\hline
		97-101 & Унификация неуспешна & Откат, достижение конца БЗ, резольвента пуста, завершение работы\\
		\hline
	\end{tabular}
\end{table}

\end{document}