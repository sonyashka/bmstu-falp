\documentclass[14pt,a4paper]{scrreprt}

\include{preambule.inc}

\begin{document}

\include{00-title}

\thispagestyle{empty}

\chapter{Практическое задание}

\section{Задание}

Используя хвостовую рекурсию, разработать эффективную программу, позволяющую:
\begin{enumerate}
	\item сформировать список из элементов числового списка, больших заданного значения;
	\item сформировать список из элементов, стоящих на нечетных позициях исходного списка (нумерация от 0);
	\item удалить заданный элемент из списка (один или все вхождения);
	\item преобразовать список в множество (можно использовать ранее разработанные процедуры).
\end{enumerate}
Убедиться в правильности результатов.

Для одного из вариантов вопроса и 1-го задания составить таблицу, отражающую конкретный порядок работы системы.

\section{Код программы}

\begin{lstlisting}[language=Prolog]
domains

	integerList = integer*.

predicates

	biggerList(integerList, integer, integerList).
	oddList(integerList, integerList).
	delElemFullInclude(integerList, integer, integerList).
	makeSet(integerList, integerList).

clauses

	biggerList([], _, []).
	biggerList([H|T], Num, [H|ResT]) :- H > Num, !, biggerList(T, Num, 
		ResT).
	biggerList([_|T], Num, Res) :- biggerList(T, Num, Res).
	
	oddList([], []).
	oddList([_, H|T], [H|ResT]) :- !, oddList(T, ResT).
	oddList([_|T], ResT) :- oddList(T, ResT).
	
	delElemFullInclude([], _, []).
	delElemFullInclude([H|T], Elem, [H|ResT]) :- H <> Elem, !, 
		delElemFullInclude(T, Elem, ResT).
	delElemFullInclude([_|T], Elem, Res) :- delElemFullInclude(T, Elem, 
		Res).
	
	makeSet([], []).
	makeSet([H|T], [H|ResT]) :- delElemFullInclude(T, H, Res), makeSet(Res, 
		ResT).

goal

	%biggerList([1, 7, 0, 4], 3, List).
	%oddList([1, 7, 0, 4], List).
	%delElemFullInclude([1, 6, 1, 0, 1], 1, List).
	makeSet([1, 5, 9, 1, 0, 5, 1], Set).
\end{lstlisting}

\section{Таблицы выполнения программы}

Запрос: biggerList([1, 7, 0, 4], 3, List).
\begin{table}[H]
	\begin{tabular}{|p{0.8cm\small}|p{4.7cm\small}|p{5.7cm\small}|p{4cm\small}|}	
		\hline
		№ шага & Состояния резольвенты и вывод: дальнейшие действия (почему?) & Для каких термов запускается алгоритм унификации: T1=T2 и каков результат (подстановка) & Дальнейшие действия: прямой ход или откат (почему и к чему приводит?)\\
		\hline
		1 & Резольвента: \begin{itemize}
			\item biggerList([1, 7, 0, 4], 3, List)
		\end{itemize} & biggerList([1, 7, 0, 4], 3, List) = biggerList([], \_, []). Унификация неуспешна & Прямой ход, переход к следующему предложению\\
		\hline
	\end{tabular}
\end{table}

\begin{table}[H]
	\begin{tabular}{|p{0.8cm\small}|p{4.7cm\small}|p{5.7cm\small}|p{4cm\small}|}	
		\hline
		№ шага & Состояния резольвенты и вывод: дальнейшие действия (почему?) & Для каких термов запускается алгоритм унификации: T1=T2 и каков результат (подстановка) & Дальнейшие действия: прямой ход или откат (почему и к чему приводит?)\\
		\hline
		2 & Резольвента: \begin{itemize} \item 1 > 3, \item !, \item biggerList([7, 0, 4], 3, ResT), \item ! \end{itemize} & biggerList([1, 7, 0, 4], 3, List) = biggerList([H|T], Num, [H|ResT]). Унификация успешна\linebreak \textbf{Подстановка:} \{H=1, T=[7, 0, 4], Num=3, List=[1|ResT]\} & Прямой ход, решение цели резольвенты\\
		\hline
		3 & Резольвента: \begin{itemize} \item !, \item biggerList([7, 0, 4], 3, ResT), \item ! \end{itemize} & 1 > 3. Ложь & Откат, переход к следующему относительно 2 предложению\\
		\hline
		4 & Резольвента: \begin{itemize} \item biggerList([7, 0, 4], 3, Res) \end{itemize} & biggerList([1, 7, 0, 4], 3, List) = biggerList([\_|T], Num, Res). Унификация успешна\linebreak \textbf{Подстановка:} \{T=[7, 0, 4], Num=3, List=Res\} & Прямой ход, решение цели резольвенты\\
		\hline
		5 & Резольвента: \begin{itemize} \item biggerList([7, 0, 4], 3, Res) \end{itemize} & biggerList([7, 0, 4], 3, Res) = biggerList([], \_, []). Унификация неуспешна & Прямой ход, переход к следующему предложению\\
		\hline
		6 & Резольвента: \begin{itemize} \item 7 > 3, \item !, \item biggerList([0, 4], 3, ResT), \item ! \end{itemize} & biggerList([7, 0, 4], 3, Res) = biggerList([H|T], Num, [H|ResT]). Унификация успешна\linebreak \textbf{Подстановка:} \{T=[7, 0, 4], Num=3, List=Res, H=7, T=[0, 4], Num=3, Res=[7|ResT]\} & Прямой ход, решение цели резольвенты\\
		\hline
		7 & Резольвента: \begin{itemize} \item !, \item biggerList([0, 4], 3, ResT), \item ! \end{itemize} & 7 > 3. Правда & Прямой ход, решение цели резольвенты\\
		\hline
		8 & Резольвента: \begin{itemize} \item biggerList([0, 4], 3, ResT), \item ! \end{itemize} & !. Отсечение 6, 7 & Прямой ход, решение цели резольвенты\\
		\hline
	\end{tabular}
\end{table}

\begin{table}[H]
	\begin{tabular}{|p{0.8cm\small}|p{4.7cm\small}|p{5.7cm\small}|p{4cm\small}|}	
		\hline
		№ шага & Состояния резольвенты и вывод: дальнейшие действия (почему?) & Для каких термов запускается алгоритм унификации: T1=T2 и каков результат (подстановка) & Дальнейшие действия: прямой ход или откат (почему и к чему приводит?)\\
		\hline
		9 & Резольвента: \begin{itemize} \item biggerList([0, 4], 3, ResT), \item ! \end{itemize} & biggerList([0, 4], 3, ResT) = biggerList([], \_, []). Унификация неуспешна & Прямой ход, переход к следующему предложению\\
		\hline
		10 & Резольвента: \begin{itemize}  \item 0 > 3, \item !, \item biggerList([4], 3, ResT), \item !, \item ! \end{itemize} & biggerList([0, 4], 3, ResT) = biggerList([H|T], Num, [H|ResT]). Унификация успешна\linebreak \textbf{Подстановка:} \{..., H=0, T=[4], Num=3, ResT=[0|ResT]\} & Прямой ход, решение цели резольвенты\\
		\hline
		11 & Резольвента: \begin{itemize}\item !, \item biggerList([4], 3, ResT), \item !, \item ! \end{itemize} & 0 > 3. Ложь & Откат, переход к следующему относительно 10 предложению\\
		\hline
		12 & Резольвента: \begin{itemize} \item biggerList([4], 3, Res), \item ! \end{itemize} & biggerList([0, 4], 3, ResT) = biggerList([\_|T], Num, Res). Унификация успешна\linebreak \textbf{Подстановка:} \{..., T=[4], Num=3, ResT=Res\} & Прямой ход, решение цели резольвенты\\
		\hline
		13 & Резольвента: \begin{itemize} \item biggerList([4], 3, Res), \item ! \end{itemize} & biggerList([4], 3, Res) = biggerList([], \_, []). Унификация неуспешна & Прямой ход, переход к следующему предложению\\
		\hline
		14 & Резольвента: \begin{itemize} \item 4 > 3, \item !, \item biggerList([], 3, ResT), \item !, \item ! \end{itemize} & biggerList([4], 3, Res) = biggerList([H|T], Num, [H|ResT]). Унификация успешна\linebreak \textbf{Подстановка:} \{..., H=4, T=[], Num=3, Res=[4|ResT]\} & Прямой ход, решение цели резольвенты\\
		\hline
	\end{tabular}
\end{table}

\begin{table}[H]
	\begin{tabular}{|p{0.8cm\small}|p{4.7cm\small}|p{5.7cm\small}|p{4cm\small}|}	
		\hline
		№ шага & Состояния резольвенты и вывод: дальнейшие действия (почему?) & Для каких термов запускается алгоритм унификации: T1=T2 и каков результат (подстановка) & Дальнейшие действия: прямой ход или откат (почему и к чему приводит?)\\
		\hline
		15 & Резольвента: \begin{itemize} \item !, \item biggerList([], 3, ResT), \item !, \item ! \end{itemize} & 4 > 3. Правда & Прямой ход, решение цели резольвенты\\
		\hline
		16 & Резольвента: \begin{itemize} \item biggerList([], 3, ResT), \item !, \item ! \end{itemize} & !. Отсечение 14, 15 & Прямой ход, решение цели резольвенты\\
		\hline
		17 & Резольвента: \begin{itemize} \item !, \item ! \end{itemize} & biggerList([], 3, ResT) = biggerList([], \_, []). Унификация успешна\linebreak \textbf{Подстановка:} \{..., H=4, T=[], Num=3, Res=[4|[]], ResT=[]\} & Прямой ход, решение цели резольвенты\\
		\hline
		18 & Резольвента: \begin{itemize} \item ! \end{itemize} & !. Отсечение 14-16 & Откат, переход к шагу 14\\
		\hline
		19 & Резольвента: пуста\linebreak \textbf{Вывод:} List=[7, 4] & !. Отсечение 6-13 & Откат, переход к шагу 6, завершение работы\\
		\hline
	\end{tabular}
\end{table}
		
\end{document}