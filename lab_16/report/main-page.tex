\documentclass[14pt,a4paper]{scrreprt}

\include{preambule.inc}

\begin{document}

\include{00-title}

\thispagestyle{empty}

\chapter{Практическое задание}

\section{Задание}

Используя хвостовую рекурсию, разработать программу, позволяющую найти:
\begin{itemize}
	\item $n!$;
	\item $n$-е число Фибоначчи.
\end{itemize}
Убедиться в правильности результатов.

Для одного из вариантов вопроса и каждого задания составить таблицу, отражающую конкретный порядок работы системы:

\section{Код программы}

\begin{lstlisting}[language=Prolog]
predicates

	factorialIter(integer, integer, integer).
	factorial(integer, integer).
	fibonIter(integer, integer, integer, integer).
	fibon(integer, integer).

clauses

	factorialIter(N, Res, IterRes) :- N = 0, !, Res = IterRes.
	factorialIter(N, Res, IterRes) :- N > 0, NewN = N - 1, 
		NewIterRes = N * IterRes,
factorialIter(NewN, Res, NewIterRes). 
	factorial(N, Res) :- factorialIter(N, Res, 1).
	
	fibonIter(N, Res, FNum, _) :- N = 0, !, Res = FNum.
	fibonIter(N, Res, _, SNum) :- N = 1, !, Res = SNum.
	fibonIter(N, Res, FNum, SNum) :- N > 1, NewN = N - 1, 
		NewFNum = SNum, NewSNUm = FNum + SNum, 
	fibonIter(NewN, Res, NewFNum, NewSNum).
	fibon(N, Res) :- fibonIter(N, Res, 0, 1).

goal

	factorial(4, Res).
	%fibon(7, Res).
\end{lstlisting}

\section{Таблицы выполнения программы}

Запрос для задания factorial(3, Res).
\begin{table}[H]
	\begin{tabular}{|p{0.8cm\small}|p{4.7cm\small}|p{5.7cm\small}|p{4cm\small}|}	
		\hline
		№ шага & Состояния резольвенты и вывод: дальнейшие действия (почему?) & Для каких термов запускается алгоритм унификации: T1=T2 и каков результат (подстановка) & Дальнейшие действия: прямой ход или откат (почему и к чему приводит?)\\
		\hline
		1 & Резольвента:\begin{itemize} \item factorial(3, Res)  \end{itemize} & factorial(3, Res) = factorialIter(N, Res, IterRes). Унификация неуспешна & Прямой ход, переход к следующему предложению \\
		\hline
		2 & ... & ... & ...\\
		\hline
		3 & Резольвента:\begin{itemize} \item factorialIter(3, Res, 1) \end{itemize} & factorial(3, Res) = factorial(N, Res). Унификация успешна\linebreak \textbf{Подстановка:} \{N=3, Res=Res\} & Прямой ход, решение цели резольвенты \\
		\hline
		4 & Резольвента:\begin{itemize} \item  3 = 0, \item !, \item Res = 1, \item ! \end{itemize} & factorialIter(3, Res, 1) = factorialIter(N, Res, IterRes). Унификация успешна\linebreak \textbf{Подстановка:} \{N=3, Res=Res, IterRes=1\} & Прямой ход, решение цели резольвенты\\
		\hline
		5 & Резольвента:\begin{itemize} \item !, \item Res = 1, \item ! \end{itemize} & 3 = 0. Ложь & Откат, переход к следующему шагу относительно 4\\
		\hline
	\end{tabular}
\end{table}

\begin{table}[H]
	\begin{tabular}{|p{0.8cm\small}|p{4.7cm\small}|p{5.7cm\small}|p{4cm\small}|}	
		\hline
		№ шага & Состояния резольвенты и вывод: дальнейшие действия (почему?) & Для каких термов запускается алгоритм унификации: T1=T2 и каков результат (подстановка) & Дальнейшие действия: прямой ход или откат (почему и к чему приводит?)\\
		\hline
		6 & Резольвента:\begin{itemize} \item 3 > 0, \item NewN = 3 - 1, 
			\item NewIterRes = 3 * 1,
\item factorialIter(NewN, Res, NewIterRes) \end{itemize} & factorialIter(3, Res, 1) = factorialIter(N, Res, IterRes). Унификация успешна\linebreak \textbf{Подстановка:} \{N=3, Res=Res, IterRes=1\} & Прямой ход, решение цели резольвенты\\
		\hline
		7 & Резольвента:\begin{itemize} \item NewN = 3 - 1, 
		\item NewIterRes = 3 * 1,
\item factorialIter(NewN, Res, NewIterRes) \end{itemize} & 3 > 0. Правда & Прямой ход, решение цели резольвенты\\
	 	\hline
	 	8 & Резольвента:\begin{itemize} \item NewIterRes = 3 * 1,
\item factorialIter(2, Res, NewIterRes) \end{itemize} & NewN = 3 - 1.\linebreak \textbf{Подстановка:} \{N=3, Res=Res, IterRes=1, NewN=2\} & Прямой ход, решение цели резольвенты\\
	 	\hline
	 	9 & Резольвента:\begin{itemize} \item factorialIter(2, Res, 3) \end{itemize} & NewIterRes = 3 * 1.\linebreak \textbf{Подстановка:} \{N=3, Res=Res, IterRes=1, NewN=2, NewIterRes=3\} & Прямой ход, решение цели резольвенты\\
		\hline
		10 & Резольвента:\begin{itemize} \item  2 = 0, \item !, \item Res = 3, \item ! \end{itemize} & factorialIter(2, Res, 3) = factorialIter(N, Res, IterRes). Унификация успешна\linebreak \textbf{Подстановка:} \{N=3, Res=Res, IterRes=1, NewN=2, NewIterRes=3, N=2, Res=Res, IterRes=3\} & Прямой ход, решение цели резольвенты\\
		\hline
	\end{tabular}
\end{table}

\begin{table}[H]
	\begin{tabular}{|p{0.8cm\small}|p{4.7cm\small}|p{5.7cm\small}|p{4cm\small}|}	
		\hline
		№ шага & Состояния резольвенты и вывод: дальнейшие действия (почему?) & Для каких термов запускается алгоритм унификации: T1=T2 и каков результат (подстановка) & Дальнейшие действия: прямой ход или откат (почему и к чему приводит?)\\
		\hline
		11 & Резольвента:\begin{itemize}\item !, \item Res = 3, \item ! \end{itemize} & 2 = 0. Ложь & Откат, переход к следующему шагу относительно 10\\
		\hline
		12 & Резольвента:\begin{itemize} \item 2 > 0, \item NewN = 2 - 1, 
		\item NewIterRes = 2 * 3,
\item factorialIter(NewN, Res, NewIterRes)\end{itemize} & factorialIter(2, Res, 3) = factorialIter(N, Res, IterRes). Унификация успешна\linebreak \textbf{Подстановка:} \{N=3, Res=Res, IterRes=1, NewN=2, NewIterRes=3, N=2, Res=Res, IterRes=3\} & Прямой ход, решение цели резольвенты\\
		\hline
		13 & Резольвента:\begin{itemize} \item NewN = 2 - 1, 
			\item NewIterRes = 2 * 3,
\item factorialIter(NewN, Res, NewIterRes)\end{itemize} & 2 > 0. Правда & Прямой ход, решение цели резольвенты\\
		\hline
		14 & Резольвента:\begin{itemize} \item NewIterRes = 2 * 3,
\item factorialIter(1, Res, NewIterRes)\end{itemize} & NewN = 2 - 1.\linebreak \textbf{Подстановка:} \{N=3, Res=Res, IterRes=1, NewN=2, NewIterRes=3, N=2, Res=Res, IterRes=3, NewN=1\} & Прямой ход, решение цели резольвенты\\
		\hline
		15 & Резольвента:\begin{itemize} \item factorialIter(1, Res, 6)\end{itemize} & NewIterRes = 2 * 3.\linebreak \textbf{Подстановка:} \{N=3, Res=Res, IterRes=1, NewN=2, NewIterRes=3, N=2, Res=Res, IterRes=3, NewN=1, NewIterRes=6\} & Прямой ход, решение цели резольвенты\\
		\hline
	\end{tabular}
\end{table}

\begin{table}[H]
	\begin{tabular}{|p{0.8cm\small}|p{4.7cm\small}|p{5.7cm\small}|p{4cm\small}|}	
		\hline
		№ шага & Состояния резольвенты и вывод: дальнейшие действия (почему?) & Для каких термов запускается алгоритм унификации: T1=T2 и каков результат (подстановка) & Дальнейшие действия: прямой ход или откат (почему и к чему приводит?)\\
		\hline
		16 & Резольвента:\begin{itemize} \item 1 = 0, \item !, \item Res = 6, \item ! \end{itemize} & factorialIter(1, Res, 6) = factorialIter(N, Res, IterRes). Унификация успешна\linebreak \textbf{Подстановка:} \{..., NewN=1, NewIterRes=6, N=1, Res=Res, IterRes=6\} & Прямой ход, решение цели резольвенты\\
		\hline
		17 & Резольвента:\begin{itemize} \item !, \item Res = 6, \item ! \end{itemize} & 1 = 0. Ложь & Откат, переход к следующему предложению относительно 16\\
		\hline
		18 & Резольвента:\begin{itemize} \item 1 > 0, \item NewN = 1 - 1, 
		\item NewIterRes = 1 * 6,
\item factorialIter(NewN, Res, NewIterRes) \end{itemize} & factorialIter(1, Res, 6) = factorialIter(N, Res, IterRes). Унификация успешна\linebreak \textbf{Подстановка:} \{..., NewN=1, NewIterRes=6, N=1, Res=Res, IterRes=6\} & Прямой ход, решение цели резольвенты\\
		\hline
		19 & Резольвента:\begin{itemize} \item NewN = 1 - 1, 
		\item NewIterRes = 1 * 6,
\item factorialIter(NewN, Res, NewIterRes) \end{itemize} & 1 > 0. Правда & Прямой ход, решение цели резольвенты\\
		\hline
		20 & Резольвента:\begin{itemize} \item NewIterRes = 1 * 6,
\item factorialIter(0, Res, NewIterRes) \end{itemize} & NewN = 1 - 1.\linebreak \textbf{Подстановка:} \{..., NewIterRes=6, N=1, Res=Res, IterRes=6, NewN=0\} & Прямой ход, решение цели резольвенты\\
		\hline
	\end{tabular}
\end{table}


\begin{table}[H]
	\begin{tabular}{|p{0.8cm\small}|p{4.7cm\small}|p{5.7cm\small}|p{4cm\small}|}	
		\hline
		№ шага & Состояния резольвенты и вывод: дальнейшие действия (почему?) & Для каких термов запускается алгоритм унификации: T1=T2 и каков результат (подстановка) & Дальнейшие действия: прямой ход или откат (почему и к чему приводит?)\\
		\hline
		21 & Резольвента:\begin{itemize} \item factorialIter(0, Res, 6) \end{itemize} & NewIterRes = 1 * 6.\linebreak \textbf{Подстановка:} \{..., N=1, Res=Res, IterRes=6, NewN=0, NewIterRes=6\} & Прямой ход, решение цели резольвенты\\
		\hline
		22 & Резольвента:\begin{itemize} \item 0 = 0, \item !, \item Res = 6, \item ! \end{itemize} & factorialIter(0, Res, 6) = factorialIter(N, Res, IterRes). Унификация успешна\linebreak \textbf{Подстановка:} \{..., NewN=1, NewIterRes=6, N=1, Res=Res, IterRes=6, N=0, Res=Res, IterRes=6\} & Прямой ход, решение цели резольвенты\\
		\hline
		23 & Резольвента:\begin{itemize} \item !, \item Res = 6 ,\item \end{itemize} & 0 = 0. Правда & Прямой ход, решение цели резольвенты\\
		\hline
		24 & Резольвента:\begin{itemize} \item Res = 6, \item ! \end{itemize} & !. Отсечение 22, 23 & Прямой ход, решение цели резольвенты\\
		\hline
		25 & Резольвента:\begin{itemize} \item ! \end{itemize} & Res = 6.\linebreak \textbf{Подстановка:} \{..., N=0, Res=6, IterRes=6\} & Прямой ход, решение цели резольвенты\\
		\hline
		26 & Резольвента: пуста\linebreak \textbf{Вывод:} Res=6 & !. Отсечение 22-25 & Откат к пункту 22, завершение работы \\
		\hline
	\end{tabular}
\end{table}

Запрос для задания fibon(3, Res).
\begin{table}[H]
	\begin{tabular}{|p{0.8cm\small}|p{4.7cm\small}|p{5.7cm\small}|p{4cm\small}|}	
		\hline
		№ шага & Состояния резольвенты и вывод: дальнейшие действия (почему?) & Для каких термов запускается алгоритм унификации: T1=T2 и каков результат (подстановка) & Дальнейшие действия: прямой ход или откат (почему и к чему приводит?)\\
		\hline
		1 & Резольвента:\begin{itemize} \item fibon(3, Res) \end{itemize} & fibon(3, Res) = factorialIter(N, Res, IterRes). Унификация неуспешна & Прямой ход, переход к следующему предложению \\
		\hline
		2-6 & ... & ... & ...\\
		\hline
	\end{tabular}
\end{table}

\begin{table}[H]
	\begin{tabular}{|p{0.8cm\small}|p{4.7cm\small}|p{5.7cm\small}|p{4cm\small}|}	
		\hline
		№ шага & Состояния резольвенты и вывод: дальнейшие действия (почему?) & Для каких термов запускается алгоритм унификации: T1=T2 и каков результат (подстановка) & Дальнейшие действия: прямой ход или откат (почему и к чему приводит?)\\
		\hline
		7 & Резольвента:\begin{itemize} \item fibonIter(3, Res, 0, 1) \end{itemize} & fibon(3, Res) = fibon(N, Res).Унификация успешна\linebreak \textbf{Подстановка:} \{N=3, Res=Res\} & Прямой ход, решение цели резольвенты\\
		\hline
		8 & Резольвента:\begin{itemize} \item fibonIter(3, Res, 0, 1) \end{itemize} & fibonIter(3, Res, 0, 1) = factorialIter(N, Res, IterRes). Унификация неуспешна & Прямой ход, переход к следующему предложению \\
		\hline
		9-10 & ... & ... & ...\\
		\hline
		11 & Резольвента:\begin{itemize} \item 3 = 0, \item !, \item Res = 0, \item ! \end{itemize} & fibonIter(3, Res, 0, 1) = fibonIter(N, Res, FNum, \_).Унификация успешна\linebreak \textbf{Подстановка:} \{N=3, Res=Res, N=3, Res=Res, FNum=0\} & Прямой ход, решение цели резольвенты\\
		\hline
		12 & Резольвента:\begin{itemize} \item !, \item Res = 0, \item ! \end{itemize} & 3 = 0. Ложь & Откат, переход к следующему предложению относительно 11\\
		\hline
		13 & Резольвента:\begin{itemize} \item 3 = 1, \item !, \item Res = 1, \item ! \end{itemize} & fibonIter(3, Res, 0, 1) = fibonIter(N, Res, \_, SNum).Унификация успешна\linebreak \textbf{Подстановка:} \{N=3, Res=Res, N=3, Res=Res, SNum=1\} & Прямой ход, решение цели резольвенты\\
		\hline
		14 & Резольвента:\begin{itemize} \item !, \item Res = 1, \item ! \end{itemize} & 3 = 1. Ложь & Откат, переход к следующему предложению относительно 13\\
		\hline
	\end{tabular}
\end{table}

\begin{table}[H]
	\begin{tabular}{|p{0.8cm\small}|p{4.7cm\small}|p{5.7cm\small}|p{4cm\small}|}	
		\hline
		№ шага & Состояния резольвенты и вывод: дальнейшие действия (почему?) & Для каких термов запускается алгоритм унификации: T1=T2 и каков результат (подстановка) & Дальнейшие действия: прямой ход или откат (почему и к чему приводит?)\\
		\hline
		15 & Резольвента:\begin{itemize} \item 3 > 1, \item NewN = 3 - 1, \item NewFNum = 1, \item NewSNUm = 0 + 1, \item fibonIter(NewN, Res, NewFNum, NewSNum) \end{itemize} & fibonIter(3, Res, 0, 1) = fibonIter(N, Res, FNum, SNum).Унификация успешна\linebreak \textbf{Подстановка:} \{N=3, Res=Res, N=3, Res=Res, FNum=0, SNum=1\} & Прямой ход, решение цели резольвенты\\
		\hline
		16 & Резольвента:\begin{itemize} \item NewN = 3 - 1, \item NewFNum = 1, \item NewSNUm = 0 + 1, \item fibonIter(NewN, Res, NewFNum, NewSNum) \end{itemize} & 3 > 1. Правда & Прямой ход, решение цели резольвенты\\
		\hline
		17 & Резольвента:\begin{itemize} \item NewFNum = 1, \item NewSNUm = 0 + 1, \item fibonIter(2, Res, NewFNum, NewSNum) \end{itemize} & NewN = 3 - 1\linebreak \textbf{Подстановка:} \{N=3, Res=Res, N=3, Res=Res, FNum=0, SNum=1, NewN=2\} & Прямой ход, решение цели резольвенты\\
		\hline
		18 & Резольвента:\begin{itemize} \item NewSNUm = 0 + 1, \item fibonIter(2, Res, 1, NewSNum) \end{itemize} & NewFNum = 1\linebreak \textbf{Подстановка:} \{N=3, Res=Res, N=3, Res=Res, FNum=0, SNum=1, NewN=2, NewFNum=1\} & Прямой ход, решение цели резольвенты\\
		\hline
		19 & Резольвента:\begin{itemize} \item fibonIter(2, Res, 1, 1) \end{itemize} & NewSNum = 0 + 1\linebreak \textbf{Подстановка:} \{N=3, Res=Res, N=3, Res=Res, FNum=0, SNum=1, NewN=2, NewFNum=1, NewSNum=1\} & Прямой ход, решение цели резольвенты\\
		\hline
	\end{tabular}
\end{table}

\begin{table}[H]
	\begin{tabular}{|p{0.8cm\small}|p{4.7cm\small}|p{5.7cm\small}|p{4cm\small}|}	
		\hline
		№ шага & Состояния резольвенты и вывод: дальнейшие действия (почему?) & Для каких термов запускается алгоритм унификации: T1=T2 и каков результат (подстановка) & Дальнейшие действия: прямой ход или откат (почему и к чему приводит?)\\
		\hline
		20-26 & Резольвента:\begin{itemize} \item fibonIter(2, Res, 1, 1) \end{itemize} & Аналогично пунктам 8-14 & \\
		\hline
		27 & Резольвента:\begin{itemize} \item 2 > 1, \item NewN = 2 - 1, \item NewFNum = 1, \item NewSNUm = 1 + 1, \item fibonIter(NewN, Res, NewFNum, NewSNum)\end{itemize}  & fibonIter(2, Res, 1, 1) = fibonIter(N, Res, FNum, SNum). Унификация успешна\linebreak \textbf{Подстановка:} \{N=3, Res=Res, N=3, Res=Res, FNum=0, SNum=1, NewN=2, NewFNum=1, NewSNum=1, N=2, Res=Res, FNum=1, SNum=1\} & Прямой ход, решение цели резольвенты\\
		\hline
		28 & Резольвента:\begin{itemize} \item NewN = 2 - 1, \item NewFNum = 1, \item NewSNUm = 1 + 1, \item fibonIter(NewN, Res, NewFNum, NewSNum)\end{itemize} & 2 > 1. Правда & Прямой ход, решение цели резольвенты\\
		\hline
		29 & Резольвента:\begin{itemize} \item NewFNum = 1, \item NewSNUm = 1 + 1, \item fibonIter(1, Res, NewFNum, NewSNum)\end{itemize} & NewN = 2 - 1\linebreak \textbf{Подстановка:} \{N=3, Res=Res, N=3, Res=Res, FNum=0, SNum=1, NewN=2, NewFNum=1, NewSNum=1, N=2, Res=Res, FNum=1, SNum=1, NewN=1\} & Прямой ход, решение цели резольвенты\\
		\hline
		30 & Резольвента:\begin{itemize} \item NewSNUm = 1 + 1, \item fibonIter(1, Res, 1, NewSNum)\end{itemize} & NewFNum = 1\linebreak \textbf{Подстановка:} \{..., N=2, Res=Res, FNum=1, SNum=1, NewN=1, NewFNum=1\} & Прямой ход, решение цели резольвенты\\
		\hline
	\end{tabular}
\end{table}

\begin{table}[H]
	\begin{tabular}{|p{0.8cm\small}|p{4.7cm\small}|p{5.7cm\small}|p{4cm\small}|}	
		\hline
		№ шага & Состояния резольвенты и вывод: дальнейшие действия (почему?) & Для каких термов запускается алгоритм унификации: T1=T2 и каков результат (подстановка) & Дальнейшие действия: прямой ход или откат (почему и к чему приводит?)\\
		\hline
		31 & Резольвента:\begin{itemize} \item fibonIter(1, Res, 1, 2)\end{itemize} & NewSNUm = 1 + 1\linebreak \textbf{Подстановка:} \{..., N=2, Res=Res, FNum=1, SNum=1, NewN=1, NewFNum=1, NewSNum=2\} & Прямой ход, решение цели резольвенты\\
		\hline
		32-36 & Резольвента:\begin{itemize} \item fibonIter(1, Res, 1, 2)\end{itemize} & Аналогично пунктам 8-12 & \\
		\hline
		37 & Резольвента:\begin{itemize} \item 1 = 1, \item !, \item Res = 2, \item !\end{itemize} & fibonIter(1, Res, 1, 2) = fibonIter(N, Res, \_, SNum). Унификация успешна\linebreak \textbf{Подстановка:} \{..., N=2, Res=Res, FNum=1, SNum=1, NewN=1, NewFNum=1, NewSNum=2, N=1, Res=Res, SNum=2\} & Прямой ход, решение цели резольвенты\\
		\hline
		38 & Резольвента:\begin{itemize} \item !, \item Res = 2, \item !\end{itemize} & 1 = 1. Правда & Прямой ход, решение цели резольвенты\\
		\hline
		39 & Резольвента:\begin{itemize} \item Res = 2, \item !\end{itemize} & !. Отсечение 37, 38 & Прямой ход, решение цели резольвенты\\
		\hline
		40 & Резольвента:\begin{itemize} \item !\end{itemize} & Res = 2\linebreak \textbf{Подстановка:} \{..., N=2, Res=2, FNum=1, SNum=1, NewN=1, NewFNum=1, NewSNum=2\} & Прямой ход, решение цели резольвенты\\
		\hline
		41 & Резольвента: пуста\linebreak \textbf{Вывод:} Res=2 & !. Отсечение 37-40 & Откат к пункту 37, завершение работы\\
		\hline
	\end{tabular}
\end{table}
		
\end{document}