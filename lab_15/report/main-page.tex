\documentclass[14pt,a4paper]{scrreprt}

\include{preambule.inc}

\begin{document}

\include{00-title}

\thispagestyle{empty}

\chapter{Практическое задание}

\section{Задание}

В одной программе написать правила, позволяющие найти:
\begin{enumerate}
	\item Максимум из двух чисел:
	\begin{itemize}
		\item без использования отсечения;
		\item с использованием отсечения;
	\end{itemize}
	\item Максимум из трех чисел:
	\begin{itemize}
		\item без использования отсечения;
		\item с использованием отсечения.
	\end{itemize}
\end{enumerate}
Убедиться в правильности результатов.

Для каждого случая пункта 2 обосновать необходимость всех условий тела. 

Для одного из вариантов вопроса и каждого варианта задания 2 составить таблицу, отражающую конкретный порядок работы системы. Т.к. резольвента хранится в виде стека, то состояние резольвенты требуется отображать в столбик: вершина -- сверху! Новый шаг нужно начинать с нового состояния резольвенты. 

\section{Код программы}

\begin{lstlisting}[language=Prolog]
domains

	num = integer.

predicates
	
	maxOfTwo(num, num, num).
	maxOfTwoCut(num, num, num).
	maxOfThree(num, num, num, num).
	maxOfThreeCut(num, num, num, num).

clauses

	maxOfTwo(Num1, Num2, Num1) :- Num1 >= Num2.
	maxOfTwo(Num1, Num2, Num2) :- Num2 >= Num1.
	
	maxOfThree(Num1, Num2, Num3, Num1) :- Num1 >= Num2, Num1 >= Num3.
	maxOfThree(Num1, Num2, Num3, Num2) :- Num2 >= Num1, Num2 >= Num3.
	maxOfThree(Num1, Num2, Num3, Num3) :- Num3 >= Num1, Num3 >= Num2.
	
	maxOfTwoCut(Num1, Num2, Num1) :- Num1 >= Num2, !.
	maxOfTwoCut(_, Num2, Num2).
	
	maxOfThreeCut(Num1, Num2, Num3, Num1) :- Num1 >= Num2, Num1 >= Num3, !.
	maxOfThreeCut(_, Num2, Num3, Num2) :- Num2 >= Num3, !.
	maxOfThreeCut(_, _, Num3, Num3).

goal

	%maxOfTwo(1, 2, Max).
	%maxOfTwo(2, 1, Max).
	%maxOfTwo(2, 2, Max).
	%maxOfTwoCut(1, 2, Max).
	%maxOfTwoCut(2, 1, Max).
	%maxOfTwoCut(2, 2, Max).
	%maxOfThree(1, 2, 3, Max).
	%maxOfThree(1, 3, 2, Max).
	%maxOfThree(3, 2, 1, Max).
	%maxOfThree(3, 2, 3, Max).
	%maxOfThreeCut(1, 2, 3, Max).
	%maxOfThreeCut(1, 3, 2, Max).
	%maxOfThreeCut(3, 2, 1, Max).
	maxOfThreeCut(3, 2, 3, Max).
\end{lstlisting}

\newpage
\section{Таблицы выполнения программы}

Запрос для задания 2.a:

maxOfThree(1, 3, 2, Max).

\begin{table}[H]
	\begin{tabular}{|p{0.8cm\small}|p{4.7cm\small}|p{5.7cm\small}|p{4cm\small}|}	
		\hline
		№ шага & Состояния резольвенты и вывод: дальнейшие действия (почему?) & Для каких термов запускается алгоритм унификации: T1=T2 и каков результат (подстановка) & Дальнейшие действия: прямой ход или откат (почему и к чему приводит?)\\
		\hline
		1 & Резольвента:\begin{itemize} \item maxOfThree(1, 3, 2, Max) \end{itemize} & maxOfThree(1, 3, 2, Max) = maxOfTwo(Num1, Num2, Num1). Унификация неуспешна & Прямой ход, переход к следующему приложению\\
		\hline
		2 & ... & ... & ... \\
		\hline
		3 & Резольвента:\begin{itemize} \item 1 >= 2, \item 1 >= 3 \end{itemize}& maxOfThree(1, 3, 2, Max) = maxOfThree(Num1, Num2, Num3, Num1). Унификация успешна\linebreak \textbf{Подстановка:} \{Num1=1, Num2=3, Num3=2, Max=Num1\} & Прямой ход, решение цели резольвенты (1 >= 2) \\
		\hline
		4 & Резольвента:\begin{itemize}\item maxOfThree(1, 3, 2, Max) \end{itemize}& 1 >= 2. Ложь & Откат, переход к следующему шагу относительно 3\\
		\hline
		5 & Резольвента:\begin{itemize} \item 3 >= 2, \item 3 >= 1 \end{itemize} & maxOfThree(1, 3, 2, Max) = maxOfThree(Num1, Num2, Num3, Num2). Унификация успешна\linebreak \textbf{Подстановка:} \{Num1=1, Num2=3, Num3=2, Max=Num2\} & Прямой ход, решение цели резольвенты (3 >= 2) \\
		\hline
		6 & Резольвента:\begin{itemize}\item 3 >= 1 \end{itemize} & 3 >= 2. Правда & Прямой ход, решение цели резольвенты (3 >= 1)\\
		\hline
		7 & Резольвента: пуста\linebreak\textbf{Вывод:} Max=3 & 3 >= 1. Правда\linebreak \textbf{Подстановка:} \{Num1=1, Num2=3, Num3=2, Max=3\} & Откат, переход к следующему шагу относительно 5\\
		\hline
		8 & Резольвента:\begin{itemize} \item 2 >= 3, \item 2 >= 1 \end{itemize} & maxOfThree(1, 3, 2, Max) = maxOfThree(Num1, Num2, Num3, Num3). Унификация успешна\linebreak \textbf{Подстановка:} \{Num1=1, Num2=3, Num3=2, Max=Num3\} & Прямой ход, решение цели резольвенты (2 >= 3) \\
		\hline
	\end{tabular}
\end{table}

\begin{table}[H]
	\begin{tabular}{|p{0.8cm\small}|p{4.7cm\small}|p{5.7cm\small}|p{4cm\small}|}	
		\hline
		№ шага & Состояния резольвенты и вывод: дальнейшие действия (почему?) & Для каких термов запускается алгоритм унификации: T1=T2 и каков результат (подстановка) & Дальнейшие действия: прямой ход или откат (почему и к чему приводит?)\\
		\hline
		9 & Резольвента:\begin{itemize}\item 2 >= 1 \end{itemize} & 2 >= 1. Ложь & Откат, переход к следующему шагу относительно 8\\
		\hline
		10 & Резольвента:\begin{itemize}\item maxOfThree(1, 3, 2, Max) \end{itemize}& maxOfThree(1, 3, 2, Max) = maxOfTwoCut(Num1, Num2, Num1). Унификация неуспешна & Прямой ход, переход к следующему предложению\\
		\hline
		11-14 & Резольвента:\begin{itemize}\item maxOfThree(1, 3, 2, Max) \end{itemize} & Унификация неуспешна & Откат, достижение конца БЗ, завершение работы\\
		\hline
	\end{tabular}
\end{table}

Запрос для задания 2.a:

maxOfThreeCut(1, 3, 2, Max).

\begin{table}[H]
	\begin{tabular}{|p{0.8cm\small}|p{4.7cm\small}|p{5.7cm\small}|p{4cm\small}|}	
		\hline
		№ шага & Состояния резольвенты и вывод: дальнейшие действия (почему?) & Для каких термов запускается алгоритм унификации: T1=T2 и каков результат (подстановка) & Дальнейшие действия: прямой ход или откат (почему и к чему приводит?)\\
		\hline
		1 & Резольвента:\begin{itemize}\item maxOfThreeCut(1, 3, 2, Max) \end{itemize} & maxOfThreeCut(1, 3, 2, Max) = maxOfTwo(Num1, Num2, Num1). Унификация неуспешна & Прямой ход, переход к следующему предложению \\
		\hline
		2-7 & ... & ... & ...\\
		\hline
		8 & Резольвента:\begin{itemize}\item 1 >= 2, \item 1 >= 3, \item ! \end{itemize} & maxOfThreeCut(1, 3, 2, Max) = maxOfThreeCut(Num1, Num2, Num3, Num1). Унификация успешна\linebreak \textbf{Подстановка:} \{Num1=1, Num2=3, Num3=2, Max=Num1\}& Прямой ход, решение цели резольвенты (1 >= 2)\\
		\hline
		9 & Резольвента:\begin{itemize}\item 1 >= 3, \item ! \end{itemize} & 1 >= 2. Ложь & Откат, переход к следующему шагу относительно 8\\
		\hline
	\end{tabular}
\end{table}

\begin{table}[H]
	\begin{tabular}{|p{0.8cm\small}|p{4.7cm\small}|p{5.7cm\small}|p{4cm\small}|}	
		\hline
		№ шага & Состояния резольвенты и вывод: дальнейшие действия (почему?) & Для каких термов запускается алгоритм унификации: T1=T2 и каков результат (подстановка) & Дальнейшие действия: прямой ход или откат (почему и к чему приводит?)\\
		\hline
		10 & Резольвента:\begin{itemize}\item 3 >= 2, \item 3 >= 1, \item ! \end{itemize} & maxOfThreeCut(1, 3, 2, Max) = maxOfThreeCut(Num1, Num2, Num3, Num2). Унификация успешна\linebreak \textbf{Подстановка:} \{Num1=1, Num2=3, Num3=2, Max=Num2\}& Прямой ход, решение цели резольвенты (1 >= 2)\\
		\hline
		11 & Резольвента:\begin{itemize}\item 3 >= 1, \item ! \end{itemize} & 3 >= 2. Правда & Прямой ход, решение цели резольвенты (3 >= 1)\\
		\hline
		12 & Резольвента:\begin{itemize}\item ! \end{itemize} & 3 >= 1. Правда & Прямой ход, решение цели резольвенты (!)\\
		\hline
		13 & Резольвента: пуста\linebreak \textbf{Вывод:} Max=3 & Операция отсечения & Откат к пункту 10, завершение работы\\
		\hline
	\end{tabular}
\end{table}
		
\end{document}